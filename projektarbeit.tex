%&encoding=UTF-8 Unicode
%
%
\documentclass[a4paper,titlepage=true,twoside]{scrartcl}

%% TODO: 
%% - Synonyme: Routine - Prozedur - Funktion - Methode -
%% - Beispiel: ``Die Qt-Programmiersprache ...''
%% - Verbraucherzentrale darf weiter die Präposition "wegen" verwenden
%%   http://www.golem.de/0912/71807.html
%% - Kürzen:
%%  Wenn man sich also entschieden haben sollte, dass man auf ein Wacom Grafik-Tablett setzen möchte
%%  => Hat man sich/Haben Sie sich für ein Wacom Grafik-Tablett entschieden, ...
%%
%% - Der haushohe Felsblock war am Montagabend kurz vor 20.00 Uhr ohne Vorwarnung 
%%   auf das Anwesen der vierköpfigen Familie gestürzt.

% Ich schreibe in UTF-8, daher:
\usepackage[utf8]{inputenc}

% Verwenden von Schriftarten in der T1-Kodierung
\usepackage[T1]{fontenc}

% 
% \usepackage{charter}
% \usepackage{palatino}

% Dokument ist in Deutsch geschrieben, daher:
\usepackage[ngerman]{babel}

% 
\usepackage{textcomp}

% Farben:
\usepackage{color}
%
%\usepackage{theorem}

% Erweiterte Referenzen:
\usepackage[ngerman]{varioref}

% Verbesserte Trennung und Anzeige von URLs
\usepackage{url}

% Zum Nummerieren von Zeilen:
\usepackage{lineno}

% If-then-else Abfrage
\usepackage{ifthen}
%\usepackage{eqexam}

\usepackage{longtable}

% Mikrotypografische Verbesserungen:
%\usepackage{fontspec}

%\usepackage{charter}

%\renewcommand{\rmdefault}{bch}


% Als letztes(!) Paket einbinden:
\usepackage[%
	pdftitle={Wie schreibe ich eine Projektarbeit?},%  Titel des PDF Dokuments.
	pdfauthor={Thomas Schraitle},%                     Autor des PDF Dokuments.
	pdfsubject={Projektarbeit},%                       Thema des PDF Dokuments.
	pdfcreator={Thomas Schraitle},%                    Erzeuger des PDF Dokuments.
	pdfkeywords={Projektarbeit,Deutsche Sprache},%     Schlüsselwörter für PDF.
	pdfpagemode=UseOutlines,%                          Inhaltsverzeichnis anzeigen beim öffnen
	pdfdisplaydoctitle=true,%                          Dokumenttitel statt Dateiname anzeigen.
	pdflang=de%                                        Sprache des Dokuments.
]{hyperref}

% Titel und andere Daten für die Titelseite definieren:
\title{\hspace{0pt}\\[2ex]Wie schreibe ich eine Projektarbeit?}
\subtitle{Sprache, Struktur, Typografie}
\author{Thomas Schraitle\\\email{toms@suse.de}}
\date{\today\\Version 1.1}

%
% Selbstdefinierte Befehle:
\makeatletter
% \renewcommand{\rmdefault}{ppl}
\newcommand{\greek}[1]{\ensuremath{#1}}
\newcommand{\gquote}[1]{\glqq #1\grqq}
\newcommand{\fquote}[1]{\frqq #1\flqq}
\newcommand{\email}[1]{\ttfamily\textless #1\textgreater}
\newcommand{\itemtitle}[1]{\textbf{#1}}
\newcommand{\coltitle}[1]{\textit{#1}}
\newcommand{\latexenv}[1]{\textsf{#1}}
\newcommand{\markerror}[1]{\textcolor{red}{#1}}
\renewcommand{\@makefnmark}{\textsuperscript{\sffamily\@thefnmark}}
\renewcommand{\@makefntext}[1]{\noindent\makebox[1.8em][r]{\textsf{\@thefnmark}\hss}#1}

\newcommand{\PreserveBackslash}[1]{\let\temp=\\#1\let\\=\temp}
\makeatother
% Selbstdefinierte Farben:
\definecolor{light}{gray}{.80}

% Selbstdefinierte Theoreme:
% {\theoremstyle{break}\theorembodyfont{\sffamily}%
% \theoremheaderfont{\sffamily}%
% \newtheorem{exa}{Beispiel}[section]
% }
% 
% {\theoremstyle{plain}\theorembodyfont{\sffamily}%
% \theoremheaderfont{\sffamily\bfseries}%
% \newtheorem{merksatz}{Merksatz}[section]
% }

\newcounter{Example}%[section]
\newcommand{\Exampletitle}{\textbf{Beispiel~\theExample}}
\newenvironment{Example}[1][]%
{%
\par\smallskip%
\setlength{\parindent}{0pt}%
\refstepcounter{Example}%
%\begin{sffamily}%
\begin{trivlist}\item[]\sffamily%
\Exampletitle%
\ifthenelse{\equal{#1}{}}%
{\par\nopagebreak}%
{:\hspace{0.75em}#1\par\nopagebreak}%
}%
{\end{trivlist}%
%\end{sffamily}%
\smallskip%
%\vspace{1.5em}
}

\newcounter{merksatz}%[Example]
\newcommand{\merksatztitle}{\textbf{Tipp~\themerksatz}}
\newenvironment{merksatz}[1][]%
{\par\smallskip%
\setlength{\parindent}{0pt}%
\refstepcounter{merksatz}%
\begin{trivlist}\item[]\sffamily%
\merksatztitle:\hspace{0.25em}
}%
{\end{trivlist}\smallskip}


% Umgebung für Zitate
\newsavebox{\Epigraph}
\newenvironment{epigraph}[1]%
{\sbox{\Epigraph}{\textit{#1}}
\begin{quote}}%
{%\\*\hspace*{\fill}\nolinebreak\hspace*{\fill}%
\\*\hspace{0pt}---\usebox{\Epigraph}%
\end{quote}}



% ###################################################
\begin{document}
\maketitle
\tableofcontents

\section*{Vorwort}
\addcontentsline{toc}{section}{Vorwort}
Dieses Dokument soll dir helfen, oft gemachte Fehler zu vermeiden.
Es fasst die wichtigsten Punkte zusammen, dennoch empfehle ich, 
ergänzende Literatur zu verwenden. Dieses Dokument kann
lediglich eine \emph{Übersicht} sein, nie eine komplette Referenz.

Herzlichen Dank an Sandra Farrell, Tanja Roth und Jürgen Weigert
für die Korrekturen und Verbesserungsvorschläge.

\vspace{1.5em}
\noindent Thomas Schraitle, \today


\newpage
%%
%% ----------------------------------------------------------
%%
\part{Sprache}
\begin{epigraph}{Isaac Goldberg}
 (Technical) communication is to write and to say\\
 The geekiest things in the simplest way.
\end{epigraph}

\section{Vergleich zwischen Schreiben und Programmieren}
Schreiben und Programmieren haben viele Gemeinsamkeiten (nach \cite{bib.rosenberg}):

\begin{small}
\begin{itemize}
 \item \itemtitle{Erfolgreiche Projekte verlangen gründliche Planung.} \\
  Umfangreiche Projekte benötigen eine sorgfältige Planung bevor eine Zeile Code
geschrieben wird. Entsprechend müssen bei großen Dokumenten Rahmenbedingungen 
festgelegt werden.
 \item \itemtitle{Erfolgreiche Projekte erfordern Prüfungen.} \\
Gute Programmierer überprüfen ihren Code auf Fehler. Entsprechend suchen 
Schreiber Rückmeldungen, um ihre Dokumentation zu verbessern.
 \item \itemtitle{Erfolgreiche Projekte benötigen Iterationen.} \\
Egal wie gut geplant wurde, um Perfektion zu erreichen erfordert Programmierung 
immer ein gewisses Maß an Wiederholung. Ein altes Sprichwort der Dokumentation
lautet: \gquote{Schreiben ist Neuschreiben}.
 \item \itemtitle{Das wichtigste Prinzip ist Sparsamkeit.} \\
Sowohl beim Programmieren als auch beim Schreiben ist einfacher besser.
 \item \itemtitle{Sparsamkeit zu erreichen ist schwierig} \\
Eleganz hat ihren Preis: Einen Algorithmus in 10 Zeilen zu programmieren dauert 
normalerweise länger als in 20 Zeilen. Entsprechend ist es schwieriger eine
komplexe Idee auf 10 Seiten zu beschreiben als auf 20 Seiten.
\end{itemize}
\end{small}



% \section{Allgemeines und Spezielles zum Schreiben}\label{sec.general}
% Für gewöhnlich stellt sich einem Autor beim Schreiben die folgenden Fragen:
% 
% %%
% \begin{description}
%  \item[Wer liest es?] \emph{Zielgruppe}\\
% %  Einsteiger, Anwender, Entwickler, Rentner, Kinder, Umsteiger, \ldots
%   Das Wichtigste beim Schreiben ist die Zielgruppe. Denn für dafür schreibst du.
%   Von wem wird dein Text gelesen? Ist es ein Einsteiger, Anwender, Entwickler,
%   Rentner, Kind oder Umsteiger? Ist deine Zielgruppe einheitlich oder uneinheitlich?
%   Welche Erwartungen hat deine Zielgruppe an den Text? Ist der Empfänger der 
%   Botschaft unbekannt, \gquote{trommelst du vor der falschen Hütte}. 
% % \begin{itemize}
% %  \item Ist die Zielgruppe homogen oder uneinheitlich? \\
% %   Beispiel: \gquote{Linux for Dummies} = Einsteigerbuch für Linuxinteressierte
% %  \item Wie ist der Bildungsstand? Hauptschule oder Hochschule? \\
% %   Beispiel: Bildzeitung, Fachbuch für organische Chemie
% %  \item Hat die Zielgruppe bestimmte Merkmale oder Einschränkungen? \\
% %   Beispiel: Kinderbücher
% %  \item Welche Erwartungen hat deine Zielgruppe? \\
% %   Beispiel: Unterhaltung, Beratung, \ldots
% % \end{itemize}
% 
%  \item[Weshalb wird dein Text gelesen?] \emph{Motivation}\\
% % Lernen, Nachschlagen, Informieren, Erinnern, Ausprobieren, \ldots
%   Was spricht den Leser an, deinen Text zu lesen? Wie verwendet der Leser
%   deinen Text: Um zu lernen, nachzuschlagen oder sich zu informieren?
% 
%  \item[Was enthält es?] \emph{Inhalt und Struktur}\\
% %  Abschnitte, Tabellen, Abbildungen, Grafiken, \ldots
%   Struktur heisst, deine Gedanken ordnen. Romane haben meist eine sehr flache Struktur,
%   Begriffe in Lexika sind nach Buchstaben sortiert und Magazine enthalten Artikel
%   mit Abschnitten. Sie alle erfüllen ihren Zweck: Das Geschriebene erkennen, 
%   einordnen und wiederfinden.
% 
%  \item[Wie wird es gelesen?] \emph{Leseart}\\
% %  linear (von vorn bis hinten), konsultierend (Lexika), 
% %  informierend (Magazin, Zeitschriften), \ldots
%   Ein Roman wird von vorn bis hinten gelesen, ein Lexikon nur dann, wenn etwas
%   benötigt wird und in einem Magazin blättern wir und lesen, wenn uns ein
%   Artikel interessiert. Wie wird dein Text gelesen?
% 
%  \item[Wie wird es dargestellt?] \emph{Layout und Typografie}\\
% %  Satzspiegel, Aufteilung der Ränder, Schriftarten, Auszeichnungen, \ldots
%   Layout und Typografie ist nie Selbstzweck. Ungünstige Typografie verursacht 
%   \gquote{Geräusche beim Lesen}: der Inhalt deiner Arbeit wird entweder 
%   missverstanden oder ins Gegenteil verkehrt. Daher muss sich die Typografie
%   dem Text anpassen und nicht umgekehrt. Für die meisten Texte heisst 
%   das oberste Prinzip \emph{Lesbarkeit}, Ausnahme ist die Werbung. 
%   Layout und Typografie sind \emph{Dienst am Leser}.
%   Die beste Typografie ist die, welche unauffällig ist.
% 
% % \item[Womit wird es gelesen?] \emph{Medium}\\
% %  gedrucktes Buch, am Bildschirm, mit einem E-Book-Lesegerät, \ldots
% \end{description}


\section{Zielgruppe kennen}
Wer liest deine Projektarbeit? Es sind IHK-Prüfer die meist ein oberflächliches Wissen von
Computern haben: Bäckermeister, Musiklehrer, Maurer oder Buchhalter. 
Deine Projektarbeit muss sowohl von einem Kollegen als auch von einem fachfremden Leser
verstanden werden. Darum: Schreibe nicht für den Kollegen vom Fach, sondern für den Bäckermeister.


\section{Wie schreibe ich verständlich?}
Ob ein Leser einen Satz versteht, hängt von vier 
Faktoren ab \cite{bib.baumert}:

\begin{description}
 \item[Bildung] Sie entscheidet, wie ein Text aussehen muss oder darf, 
 damit Leser ihn verstehen können.
 \item[Sprachwissen] Muttersprachler oder Deutsch als Fremdsprache? 
Wer Deutsch nicht als Muttersprache spricht, kennt oft weniger Wörter als Muttersprachler.
 \item[Fachkenntnisse] Liest ein Fachmann oder Laie? 
 \item[Lesealter] Je jünger desto geringer der Wortschatz.
\end{description}

Aus den genannten Punkten ergeben sich die folgenden Tipps, wie ein Text
verständlich wird (aus \cite{bib.rosenberg}):

\begin{itemize}
 \item Schreibe zielgruppenspezifisch.
 \item Schreibe eindeutig.
 \item Schreibe prägnant.
 \item Hilf dem Leser.
\end{itemize}


% Das Theorem des technischen Schreibens:
% 
% \begin{itemize}
%  \item Halte deine Sätze kurz indem du unnötige Wörter entfernst.
%  \item Schreibe Sätze hauptsächlich im Aktiv.
%  \item Entferne unnötige Sätze oder bedeutungslose Konzepte.
% \end{itemize}
% 
% Ein weiteres Theorem
% 
% \begin{itemize}
%  \item Sag deinem Leser was du gleich sagen wirst.
%  \item Sag es.
%  \item Fasse zusammen, was du gesagt hast.
% \end{itemize}


\section{Passende Wörter finden}
\begin{epigraph}{Cäsar}
 Jedes selten gehörte und ungewohnte Wort solltest du fliehen wie ein Riff.
\end{epigraph}

%In \cite{bib.baumert} wird vermutet, dass konkrete Substantive (Maus, Festplatte, 
%Bildschirm \ldots) anders im Langzeitgedächnis gespeichert sind als abstrakte 
%Wörter (Algorithmus, Toolkit, Tendenz \ldots)


\subsection{Komposita auflösen}
Komposita sind \emph{zusammengesetzte Wörter}: aus \emph{Käse} und \emph{Kuchen}
wird der \emph{Käsekuchen}, also ein Kuchen der Käse enthält. Preisfrage: Aus was 
besteht \emph{Hundekuchen}, \emph{Sandkuchen}, \emph{Baumkuchen}, 
\emph{Pustekuchen}\ldots?

\begin{Example}[Zusammengesetzte Wörter auflösen]
\begin{longtable}{lp{.45\linewidth}}
 \coltitle{Schlecht}   & \coltitle{Besser}  \\
% Autobahngegenrichtungsfahrbahnbenutzer 
Autobahnfahrstreifengegenrichtungsbenutzer
& Geisterfahrer, Falschfahrer \\
Kinematographentheater  & Kino \\
Haupteinflussfaktor       & Vor allem beeinflusst\ldots \\
Dateinamenexpandierungsoption & Option für Dateinamensexpandierung --- oder: \\
                          & Dateinamen werden aufgelöst mit der Option\ldots \\
Festplattenanschlusskabelklemme & Klemme zum Anschließen des Festplattenkabels \\
Programmausnahmefehler    & Ausnahme (oder Fehler) im Programm \\
Lizenzfreischaltungsknopf & Freischalten der Lizenz durch/mit\ldots \\
\end{longtable}
\end{Example}

\begin{merksatz}
Verwende möglichst kurze Wörter.
\end{merksatz}

\subsection{Synonyme mit Bedacht wählen}
Ein Synonym ist ein \emph{sinnverwandtes Wort}. Jedoch sind die wenigsten
Wörter wirklich austauschbar: Anlitz, Fratze, Visage, Fresse haben zwar mit
dem Begriff \emph{Gesicht} zu tun, die \emph{Stilebene} ist jedoch unterschiedlich.

Die Computertechnik nutzt eine Fülle von Fach- und Fremdwörtern für die es 
keine Synonyme gibt: Grafikkarte, Maus, regulärer Ausdruck und andere.

Auch wenn im Deutsch-Unterricht die Wiederholung verpönt war: es ist besser
ein Wort zu wiederholen als krampfhaft zu versuchen ein (unpassendes) 
Synonym zu finden. Manchmal hilft es auch, den Satz etwas umzuschreiben.

\begin{merksatz}
Verwende für ein Objekt in deinem Dokument immer den gleichen Begriff.
Beispiel: für Dialogfenster immer Dialogfenster, nicht Screen, Maske, Dialog,
Programmfenster, Hauptfenster\ldots
\end{merksatz}


\subsection{Fremdwörter bewusst einsetzen}
In folgenden Fällen sind Fremdwörter willkommen oder erlaubt:

\begin{itemize}
 \item es verständlich und treffend ist (Code, \ldots)
 \item es verständlich und auf dieser Stilebene nicht durch ein deutsches Wort zu
ersetzen ist (Routing(?), Entity, \ldots)
 \item zwar nicht allgemein verständlich, aber bisher ohne deutsche Entsprechung
  ist (Cluster)
\end{itemize}

Definiere unbekannte Wörter. Meist bietet es sich an, das innerhalb eines
Satzes zu tun und für die Erklärungen einen weiteren Satz anzufügen (siehe auch
Beispiel~\ref{ex.termdef}).
Der neue Begriff wird hierbei in \emph{Kursivschrift} gesetzt:

\begin{Example}[Definieren von Begriffen]
 Gespeichert werden die Daten auf einer \emph{Festplatte}. Eine Festplatte ist ein
magnetisches Speichermedium in der Computertechnik. [\ldots]
\end{Example}

%\begin{merksatz}
%Verwende Fremdwörter da, wo es sinnvoll ist. Definieren nicht vergessen!
%\end{merksatz}


\subsection{Anglizismen \gquote{killen}}
Ein Sonderfall des Fremdworts ist der \emph{Anglizismus}. Alexander Baumert 
schreibt in \cite{bib.baumert}:

\begin{quote}
2006 hat die Endmark GmbH zwölf englisch-deutsche Werbesprüche auf ihre
Verständlichkeit getestet. Am besten schnitt \emph{Feel the difference} ab, 
immerhin 55\,\% übersetzen diese Lösung richtig. [\ldots] 45\,\% begreifen
den zentralen Satz falsch!
\end{quote}

\begin{Example}[Fehlerhaft verstandene Werbesprüche]
\begin{tabular}{ll}
 \coltitle{Original}  & \coltitle{Wurde verstanden als\ldots}  \\
Feel the difference   & Fühle das Differential (Ford)\\
Impossible is nothing & ein imposantes Nichts (adidas) \\
Make the most of now  & mach keinen Most daraus (Vodafone) \\
Fly Euro Shuttle      & der Euro-Schüttel-Flug /\\
                      & Schüttel den Euro zum Fliegen (Air-Berlin) \\
Come in and find out & Komm herein und finde wieder heraus (Douglas) %\\
%Amazon -- and you're done --- Amazon und du bist getan
\end{tabular}
\end{Example}

Die Frage lautet: Verstehen die IHK-Prüfer den englischen Begriff? 
Oder \emph{glauben} sie ihn zu verstehen? 

Ein Anglizismus per se ist weder gut noch schlecht; Schlecht ist er dann,
wenn ohne Not ein englischer Begriff gewählt wurde, obwohl es eine passende
deutsche Übersetzung gibt. Besser ist daher \emph{Festplatte} statt \emph{harddisk},
\emph{Speicher} statt \emph{memory}, \emph{Tastatur} statt 
\emph{keyboard}\footnote{Keyboard ist im Englischen zweideutig: Es kann eine 
Computertastatur gemeint sein oder eine Klaviatur.} 
und \emph{Fehler} statt \emph{error}. 

Gibt es keinen passenden deutschsprachigen Begriff? Definiere ihn, wie
folgendes Beispiel zeigt:

\begin{Example}[Mehrere Möglichkeiten wie ein englischer Begriff definiert werden kann]\label{ex.termdef}
Formuliere einen Satz, in dem das Wort \emph{Hyperlink} vorkommt:
\begin{enumerate}
 \item Durch getrennte Hauptsätze:\\
  Webseiten enthalten \emph{Hyperlinks}. Hyperlinks sind Querverweise auf andere Seiten.
 \item Durch einen Nebensatz:\\
  Webseiten enthalten \emph{Hyperlinks}, [dies sind] Querverweise auf andere Seiten.
 \item Der \gquote{so genannt} Anhang:\\
  Webseiten enthalten Querverweise, so genannte \emph{hyperlinks}.
 \item Die \gquote{englische Klammer}:\\
  Webseiten enthalten Querverweise (engl. \emph{hyperlinks}).
\end{enumerate}
\end{Example}


\begin{merksatz}
Wenn möglich vermeide Anglizismen. Ersetze Anglizismen durch eindeutige deutschsprachige
Übersetzung. Erkläre sperrige Begriffe besser in Sätzen als in einem Wort.
\end{merksatz}


% 
% \hspace{0pt}\\
% \begin{Exercise}[title={Anglizismen},difficulty=3,label={ex.anglizismen2}]
% \Question[title={Finde Erklärungen zu folgenden englischen Begriffen}]\\
% applet, % kleine Anwendung; spez. eingebettete Funktion in komplexen Anwendungen
% API, % API, Anwendungsprogramm-Schnittstelle
% byte code, %
% clip art, % Sammlung von Grafikbildchen
% eye tracking, % 
% test case, % Testumgebung
% \end{Exercise}
% \begin{Answer}[ref={ex.anglizismen2}]
% Applet = kleine Anwendung; spez. eingebettete Funktion in komplexen Anwendungen
% API = API, Anwendungsprogramm-Schnittstelle;
% byte code = ???;
% clip art = Sammlung von Grafikbildchen;
% eye tracking = Augenverfolgung, Bewegungserfassung der Augen;
% test case =  % Testumgebung
% widget = (window + gadget)  Vorrichtung im Fenster, Dingsbums, Komponente
% \end{Answer}



\subsection{Tautologien kürzen}
Tautologie vom griechischem \greek{\tau\alpha\upsilon\tau{}o\lambda{}o\gamma\iota\alpha}
\gquote{Dasselbe-Sagen}. % \greektext{ταυτολογία}


% \hbox{}\par
%\begin{Exercise}[title={Tautologien},label={ex.tautologien},difficulty=1]
% \Question[title={Finde weitere Tautologien}]
% \begin{Answer}[ref={ex.tautologien}]%

\begin{Example}[Tautologien]
\begin{longtable}{p{.49\textwidth}p{.49\textwidth}}
binäre Ja-Nein-Abfrage & ersetzbare Platzhalter \\
unveränderliche Konstante & geschütztes DRM \\
konstantes Tupel & % in Python: Tupel sind immer konstant
   duales Zweiersystem \\ %
neu renoviert & dunkle Ahnungen\\% Gegensatz: Wissen
dicke Trossen & % Trossen sind immer dick, ansonsten wäre es Seile
   seltene Raritäten \\
wimmelnder Überfluss & schwere Verwüstungen\\%
restlos überzeugt & steile Felswände\\% Wenn die Wand aufhört steil zu sein, wird sie zum Hang
weißer Nebel & flache Kuppen\\% ansonsten wären es Spitzen
\end{longtable}
\end{Example}

Tautologien verwirren mehr als dass sie aufklären. Entfernt das Adjektiv
vor dem Hauptwort, das bringt Klarheit. Das heißt, es reicht völlig aus von
\emph{Konstante} zu schreiben ohne \emph{unveränderlich}.

%
\section{Sätze bilden}
\begin{epigraph}{Wolf Schneider}
Die dicke Muse des deutschen Satzes.
\end{epigraph}


%\subsection{Im Präsens schreiben}
%FIXME: Mehr dazu schreiben.\\
%Schreibt Projektarbeiten in der Gegenwart (Präsens).

\subsection{Verben suchen}
\begin{epigraph}{Faust I, Studierzimmer}
Am Anfang war die Tat.
\end{epigraph}

\noindent Verben sind \gquote{Tun- oder Tatwörter}, sie beschreiben eine Aktion. Jedoch gibt
es gute und schlechte Verben. Schlechte Verben sind folgende:

\begin{description}
\item[Luftwörter:] erfolgen, bewirken, bewerkstelligen;
\item[Spreizwörter:] beinhalten, vergegenwärtigen;
\item[tote Verben:] sich befinden, liegen, gehören, [es gibt];
\item[Blähverben:] aufweisen, weilen;
\end{description}

Warum du besser einen Satz mit Verben schreiben solltest als mit Hauptwörtern
(Substantive) zeigt folgendes Beispiel:


\begin{Example}[Substantiv-Stil]
Das Beißen der Briefträger erfolgt durch Hunde.\\
Die Installation des Programms erfolgt durch Einlegen der CD.
\end{Example}

Das sagt kein Mensch! Und doch wird es häufig geschrieben. Ersetze das
Hauptwort (das Beißen, die Installation) durch ein Verb (beißen, installieren):

\begin{Example}[Diesmal ohne Substantiv]
Hunde beißen Briefträger. Oder: Briefträger werden von Hunden gebissen.\\
Installieren Sie das Programm, indem Sie die CD [in Ihr Laufwerk] einlegen.
\end{Example}


\subsection{Verbklammern auflösen}
Zusammengesetzte Verben sind \gquote{an-zeigen}, \gquote{ein-binden} und noch weitere. 
Das Deutsche hat die unangenehme Eigenschaft, dass sich solche Verben auseinanderreißen
lassen. Dies führt zu einer so genannten \emph{Verbklammer}: Ein Teil des Wortes steht
am Anfang und ein anderer Teil am Ende des Satzes.

\begin{Example}[Verbklammer]
Fur den Arbeitgeber \textbf{fallen} außer Personalkosten keine weiteren Kosten \textbf{an}.\\
Das Dialogfenster \textbf{zeigt} eine Vielzahl von unterschiedlichen Optionen \textbf{an}.\\
Programm X \textbf{bindet} das Gerät unter einem neuen Namen \textbf{ein}.
\end{Example}

Wenn zwischen ersten und zweitem Teil zuviele Wörter enthalten sind, solltest du die
Verbklammer auflösen. Wieviele Wörter sind \emph{zuviele}? Falls noch Nebensätze eingefügt
wurden oder du mehr als 6 Wörter zwischen den Verbteilen hast.

Manchmal musst du die Sätze etwas umformulieren. Die vorigen Beispiele sehen verbessert so aus:

\begin{Example}[Aufgelöste Verbklammer]
Für den Arbeitgeber \textbf{entstehen} außer Personalkosten keine weiteren Kosten.\\
Das Dialogfenster \textbf{zeigt} eine Vielzahl von unterschiedlichen Optionen.\\
Wenn Programm X das Gerät \textbf{einbindet}, ist es unter einem neuen Namen sichtbar.
\end{Example}


\subsection{Modalverben eindeutig verwenden}
Dürfen, können, mögen, müssen, sollen, wollen sind Modalverben.
Das Wort \emph{modal} kommt von lateinisch \gquote{Art und Weise}. Mit den
Modalverben drückst du deine Einstellung und Wertung gegenüber dem im Verb
Ausgesagten aus.

Markiere eindeutig, ob es sich lediglich um eine Empfehlung oder ein 
unumgängliches Muss handelt.

% \let\PBS=\PreserveBackslash
% \begin{longtable}{lp{.3\textwidth}p{.4\textwidth}}
% Modalverb & Zweck & Beispiel \\
% \hline
% dürfen & Erlaubnis/Berechtigung & \raggedright
%  Die Maschine darf nur von ausgebildeten Technikern gewartet werden. \\
%  & Vermutung & \raggedright{} Der rote Schalter dürfte der richtige sein.\\
% \hline
% \pagebreak[3]
% können & Möglichkeit/Erlaubnis & \raggedright Der Monteur kann in zwei Stunden
% mit der Reparatur beginnen. \\
%  & Fähigkeit & Der Programmierer kann verschiedene Sprachen.\\
%  & Aufforderung & Kannst du mir schnell helfen?\\
% \hline
% müssen & zwingende Notwendigkeit & Das muss heute noch gemacht werden.\\
%        & Aufforderung/Befehl & \raggedright Sie müssen die Arbeit unterbrechen,
% wenn das Alarmsignal ertönt.\\
%        & sichere Vermutung & Der Mann im weißen Kittel muss der
% Projektleiter sein. \\
% \hline
% sollen & \raggedright Auftrag, Pflicht, nicht zwingende Notwendigkeit 
%   & \raggedright Die Maschine soll regelmäßig gewartet werden.\\
%        & Behauptung & Das sollen die Leistungsmerkmale der
% Oberfläche sein, das wir entwickeln.\\
% \end{longtable}


\begin{description}
\item[dürfen = Erlaubnis/Berechtigung:]\hspace{0pt}\\ 
Die Maschine darf nur von ausgebildeten Technikern gewartet werden.

\item[dürfen = Vermutung:]\hspace{0pt}\\ 
Der rote Schalter dürfte der richtige sein.

\item[können = Möglichkeit/Erlaubnis:]\hspace{0pt}\\ 
Der Monteur kann in zwei Stunden mit der Reparatur beginnen.

\item[können = Fähigkeit:]\hspace{0pt}\\ 
Der Programmierer kann verschiedene Sprachen.

\item[können = Vemutung:]\hspace{0pt}\\ 
Das kann doch nicht dein Ernst sein!

\item[können = Aufforderung:]\hspace{0pt}\\ 
Kannst du mir schnell helfen?

\item[müssen = Zwingende Notwendigkeit:]\hspace{0pt}\\
Das muss heute noch gemacht werden.

\item[müssen = Aufforderung/Befehl:]\hspace{0pt}\\
Sie müssen die Arbeit unterbrechen, wenn das Alarmsignal ertönt.

\item[müssen = Sichere Vermutung:]\hspace{0pt}\\
Der Mann im weißen Kittel muss der Projektleiter sein.

\item[sollen = Auftrag/Pflicht/nicht zwingende Notwendigkeit:]\hspace{0pt}\\
Die Maschine soll regel\-mäßig gewartet werden.

\item[sollen = Behauptung:]\hspace{0pt}\\
Das sollen die Leistungsmerkmale der Oberfläche sein, das wir entwickeln.
\end{description}


%% FIXME: Beispiel einfügen?



\subsection{Passiv sinnvoll nutzen}
Der Satz \gquote{Ihre Daten werden gelöscht} verschweigt, \emph{wer} dies tut.
Passivsätze sind problematisch, weil:

\begin{itemize}
 \item Leser länger brauchen, um einen Satz im Passiv zu verstehen,
 \item Leser den Passivsatz dafür leichter missverstehen,
 \item wichtige Informationen im Passiv in den Hintergrund treten oder völlig
 verschwinden.
\end{itemize}

Die folgenden Beispiele zeigen Sätzen im Passiv:

\begin{Example}[Sätze im Passiv]
Das Programm wird installiert. \emph{Von wem? Automatisch?}\\
Die Gebühren für die Lizenz mussten erhöht werden. \emph{Wer hat?}\\
Aus Sicherheitsgründen werden die Daten gelöscht. \emph{Warum? Wer?} \\
Im Fenster wird auf Ok geklickt, um den Prozess zu starten. \emph{Wer?}\\
Mitarbeiter wurden freigesetzt. \emph{Aha, von wem?} \\
Signale werden übertragen. \emph{Wodurch?}  \\
Produktionskapazitäten werden stillgelegt. \emph{Von wem?}
\end{Example}

Nenne Ross und Reiter und formuliere um ins Aktiv:

\begin{Example}[Vorige Sätze im Aktiv]
Installieren Sie das Programm.\\
Wir/Die Geschäftsleitung/\ldots\ mussten die Gebühren erhöhen.\\
Das Programm löscht die Daten aus Sicherheitsgründen.\\ 
Klicken Sie auf OK um den Prozess zu starten.\\
Wir/Die Geschäftsleitung/\ldots\ haben Mitarbeiter entlassen.\\
Der Transmitter überträgt Signale.\\
Firma X musste Produktionskapazitäten stilllegen, weil\ldots
\end{Example}


Während in einem Aktivsatz das Subjekt wichtig ist, steht im Passivsatz die 
Handlung im Vordergrund. Passivsätze sind zulässig, wenn:

\begin{itemize}
 \item Die handelnde Person keinen interessiert
 (\gquote{Das Museum wird um 18\,Uhr geschlossen.})
 \item Die handelnde Person unbekannt ist 
(\gquote{Neujahr wird überall gefeiert.}, \gquote{Das Haus wurde renoviert.})
 \item Aus politischen Gründen die Person verschwiegen werden soll 
(\gquote{Fehler wurden gemacht.})
 \item Die Handlung im Vordergrund stehen soll.
\end{itemize}



% \subsection{Vereinfachungen suchen}
% \gquote{Nennen Sie die Absatzwege im Gartenbau.} ---
% \gquote{Wo kann der Gärtner seine Ware verkaufen?}
% 
% \gquote{Freisetzen von Rationalisierungspotentialen im Dienstleistungsbereich} ---
% \gquote{Kündigung von Putzfrauen}


\subsection{Einschübe sprengen}
Schachtelsätze sind vergleichbar mit Funktionen, die andere Funktionen aufrufen.
Bei mehr als zwei Ebenen haben die meisten Leser Probleme.

\begin{Example}[Aus der Chemie]\label{exa.chemie}
In seiner natürlichen Form ist molekularer Stickstoff, der aus zwei 
Stickstoff-Atomen, die von einer Dreifach-Bindung zusammengehalten 
werden, besteht, eines der stabilsten Moleküle auf der Erde.
\end{Example}

Der vorige Satz hat drei Ebenen/Einschübe:

\begin{enumerate}
\item In seiner natürlichen Form ist molekularer Stickstoff [...] eines 
 der stabilsten Moleküle auf der Erde.
\item ... der aus zwei Stickstoff-Atomen [...] besteht ...
\item ... die von einer Dreifach-Bindung zusammengehalten werden ...
\end{enumerate}

Beispiel~\ref{exa.chemie} enthält zuviel Informationen. Es kostet Mühe, die
verschiedenen Ebenen zu erkennen und richtig zuzuordnen. Besser ist es, den
langen Satz in mehrere kurze Sätze aufzuteilen:

\begin{Example}[Verbesserte Version]
In seiner natürlichen Form ist molekularer Stickstoff eines der stabilsten
Moleküle auf der Erde. Er besteht aus zwei Stickstoff-Atomen, die
von einer Dreifach-Bindung zusammengehalten werden.
\end{Example}

Juristen sind die furchtbarsten Sprachpanscher wie folgendes Beispiel zeigt:

% \paragraph{}
\begin{Example}[Negativbeispiel aus dem Bürgerlichen Gesetzbuch]
Ist eine Willenserklärung nach §118 nichtig oder auf Grund der
§§119, 120 angefochten, so hat der Erklärende, wenn die Erklärung
einen anderen gegenüber abzugeben war, diesem, andernfalls jedem
% Dritten den Schaden zu ersetzen, den der andere oder der Dritte
dadurch erleidet, dass er auf die Gültigkeit der Erklärungen
vertraut, jedoch nicht über den Betrag des Interesses hinaus,
welches der andere oder der Dritte an der Gültigkeit der Erklärungen
hat. (§112, Abs. 1)
\end{Example}


% \section{Der Dreisprung}
% 
% \begin{enumerate}
% \item Sage den Leuten was du Ihnen gleich sagen wirst.
% \item Sage es Ihnen.
% \item Sage Ihnen, was du gesagt hast.
% \end{enumerate}


\subsection{Füllwörter streichen}
Füllwörter \gquote{würzen} den Text. Wie bei jedem Gewürz ist ein Zuviel schädlich
für den Genuss. Daher lautet die Regel: Streiche möglichst viele dieser
Wörter. Die folgende Liste stammt von \cite{bib.schneider01}:

\begin{small}
aber, abermals, allein, allemal, allem Anschein nach, allenfalls, allenthalten,
allerallerdings, allerdings, allesamt, allzu, also, andauerend, andererseits,
andernfalls, anscsheinend, an sich, auch, auffallend, aufs Neue, augenscheinlich,
ausdrücklich, ausgerechnet, ausnahmslos, außerdem, äußerst, beinahe, bei weitem,
bekanntlich, bereits, besonders, bestenfalls, bestimmt, bloß, dabei,
dadurch, dafür, dagegen, daher, damals, danach, dann und wann, demgegenüber,
demgemäß, demnach, denkbar, denn, dennoch, deshalb, des Öfteren, des ungeachtet,
deswegen, doch, durchaus, durchweg, eben, eigentlich, ein bisschen, einerseits,
einfach, einige, einigermßaen, einmal, ein wenig, ergo, erheblich, etliche,
etwa, etwas, fast, folgendermaßen, folglich, förmlich, fortwährend, fraglos,
freilich, ganz, ganz und gar, gänzlich, gar, gelegentlich, gemeinhin,
genau, geradezu, gewiss, gewissermaßen, glatt, gleichsam, gleichwohl, 
glücklicherweise, gottseidank, größtenteils, halt, häufig, hie und da,
hingegen, hinlänglich, höchst, höchstens, im Allgemeinen, immer, immerhin, 
immerzu, in der Tat, indessen, in diesem Zusammenhang, infolgedessen,
insbesondere, inzwischen, irgend, irgendein, irgendjemand, irgendwann,
irgendwie, irgendwo, ja, je, jedenfalls, jedoch jemals, kaum, keineswegs,
längst, lediglich, leider, letztlich, manchmal, mehrfach, mehr oder weniger,
meines Erachtens, meinetwegen, meist, meistens, meistenteils,
mindestens, mithin, mitunter, möglicherweise, möglichst, nämlich, naturgemäß,
natürlich, neuerdings, neuerlich, neulich, nichtsdestoweniger, nie, niemals, 
nun, nur, offenbar, offenkundig, offensichtlich, oft, ohnedies, ohne weiteres,
ohne Zweifel, partout, plötzlich, praktisch, quasi, recht, reichlich,
reiflich, relativ, restlos, richtiggehend, rundheraus, rundum, samt und sonders,
seltsam, schlicht, schlichtweg, schließlich, schlussendlich, schon, sehr,
selbst, selbstredend, selbstverständlich, selten, seltsamerweise, sicher,
sicherlich, so, sogar, sonst, sowieso, sowohl als auch, sozusagen, stellenweise,
stets, trotzdem, überaus, überdies, umständehalber, unbedingt, unerhört, 
ungefähr, ungemein, ungewöhnlich, ungleich, unglücklicherweise, unlängst,
unmaßgeblich, unsagbar, unsäglich, unstreitig, unzweifelhaft, vergleichsweise,
vermutlich, vielfach, vielleicht, voll, vollends, völlig, vollkommen,
vollständig, voll und gnaz, von neuem, wahrscheinlich, weidlich, weitgehend,
wenigstens, wieder, wiederum, wirklich, wohl, wohlgemerkt, womöglich,
ziemlich, zudem, zugegeben, zumeist, zusehends, zuweilen, zweifellos, zweifelsfrei,
zweifelsohne
\end{small}


\begin{Example}[Füllwörter in einem Text]
Das Programm funktioniert \emph{höchst} eigenartig. \emph{Denkbar} wäre eine falsche
Eingabe, \emph{vielleicht} \emph{nur} ein Leerzeichen.
\end{Example}

Kritikpunkte des vorigen Textes: Der Autor war sich unsicher über die Ursache des 
Problems. Wenn diese Unklarheiten beseitigt sind, kann der Text wie folgt
aussehen:

\begin{Example}[Korrektur ohne Füllwörter]
%Das Programm reagiert falsch, wenn ein Leerzeichen eingegeben wird.\\
%Besser:
Wenn ein Leerzeichen eingegeben wird, reagiert das Programm falsch.
\end{Example}

\begin{merksatz}
Überprüfe deinen Satz auf überflüssige Füllwörter.
\end{merksatz}


\subsection{Reihenfolge beachten}
Nach eins folgt zwei. %, auf A folgt B. 
Häufig wird jedoch zuerst zwei genannt wie im folgenden Beispiel:

\begin{Example}[Falsche Reihenfolge]
Um die Synchronisation auszulösen, starten Sie zuvor das Programm Foo.
\end{Example}

Es ist für den Leser wesentlich einfacher, wenn die Reihenfolge umgestellt wird:

\begin{Example}[Korrigierte Reihenfolge]
Starten Sie das Programm Foo, um die Synchronisation auszulösen.
\end{Example}

\begin{merksatz}
Schreibe deine Aufzählungen so, dass die Reihenfolge logisch ist.
\end{merksatz}


\subsection{Nein zum Neinsagen}
\begin{epigraph}{Joachim Ringelnatz}
Sicher ist, dass nichts sicher ist. Selbst das nicht.
\end{epigraph}

Laut einer amerikanischen Untersuchung in Psychology today, 9/1974 braucht der
Durchschnittsmensch 48 Prozent mehr Zeit, um eine verneinende Satzaussage zu
verstehen als eine bejahende.

Das Deutsche kennt einige Möglichkeiten nein zu sagen (nach \cite{bib.schneider01}):

\begin{enumerate}
 \item Verneinung durch Wörter
\begin{enumerate}
 \item Direkte Verneinungen: kein, nein, nicht, nichts, nie, niemals, niemand, Null,
 ohne, weder -- noch, wenig, \ldots
 \item Integrierte Verneinung: hindern (nicht zulassen), kaputt (nicht funktionsfähig),
 kaum (nicht genug), knapp (nicht genug), Mangel (das Nichtgenughaben), selten (nicht
oft genug), zweifeln (nicht glauben), \ldots
\end{enumerate}

 \item Verneinung durch Silben
 \begin{enumerate}
  \item durch Vorsilben: a- (amoralisch), ab- (ablehnen, abnormal), auf- (aufheben =
abschaffen), aus- (ausbleiben), de- (dementieren), des- (desinifzieren), dis- 
(disqualifizieren), durch- (durchkreuzen), ein- (einbüßen), ent- (entbehren, entfernen),
un- (unmäßig), ver- (verbieten), weg- (weglassen), zurück- (zurückweisen)
 \item durch Nachsilben: -frei (koffeinfrei), -leer (inhaltsleer), -los (arbeitslos)
 \end{enumerate}
\end{enumerate}


\begin{Example}[Verneinungen in Sätzen]\label{ex.verneinung}
%\vspace*{-0.5em}
\begin{enumerate}
 \item Die FSF klagt über einen \emph{zunehmenden Mangel} an Entwicklern.\\
 Besser: Die FSF klagt darüber, dass es immer weniger Entwickler gibt.

 \item Es vergeht \emph{kaum} ein Tag, an dem die Organisation \emph{keine} Verletzungen der GPL findet.\\
 %1. Es vergehen sehr viele Tage, an denen sie eine Verletzung finden.
 Besser: Jeden Tag findet die Organisation GPL-Verletzungen.
 %3. An (fast) jedem Tag findet die Organisation GPL-Verletzungen.
 % \item ... Ein Beispiel? ...
\end{enumerate}
\end{Example}

% Wenn in Beispiel~\ref{ex.verneinung} die Verneinung umformuliert wird, lauten die 
% Sätze wie folgt:
% 
% \begin{Example}[Umformulierte Verneinungen in Sätzen]
% %\vspace*{-0.5em}
% \begin{enumerate}
%  \item 
% % Die FSF klagt über einen zunehmenden Mangel an Entwicklern.
% Die FSF klagt darüber, dass es weniger Entwickler gibt.
%  \item 
% % Es vergeht kaum ein Tag, an dem die Organisation keine Verletzungen der GPL findet.
% % Es vergehen sehr viele Tage, an denen sie eine Verletzung finden.
% Jeden Tag findet die Organisation GPL-Verletzungen.
%  %3. An (fast) jedem Tag findet die Organisation GPL-Verletzungen.
%  % \item ... Ein Beispiel? ...
% \end{enumerate}
% \end{Example}


\begin{merksatz}
 Schreibe Sätze möglichst in der \gquote{Ja-Form}.
\end{merksatz}


% \begin{quote}
% Es vergeht kaum ein Tag, an dem die Organisation keine 
% Verletzungen der GPL finden würde
% \end{quote}

% \begin{Exercise}[title={Verneinende Satzaussagen},difficulty=2,label={ex.nosentences}]
% \Question[title={Schreibe folgende Sätze um}]\\
% Es vergeht kaum ein Tag, an dem die Organisation keine 
% Verletzungen der GPL finden würde.
% 
% Wenn der Rückgabewert ungleich 2 ist, springt die Funktion 
% 
% Die Uno klagt über zunehmenden Mangel an Weizen.
% 
% \end{Exercise}

%
%Korrektur:
%1. Es vergehen sehr viele Tage, an denen sie eine Verletzung finden.
%2. Jeder Tag werden GPL-Verletzungen gefunden.
%3. An (fast) jedem Tag findet die Organisation GPL-Verletzungen.
%

% Nichts genaues weiß man nicht.
% Null Bock auf gar nichts
% 
% Überfluss an Geistesmangel
% 
% AIDS-Test positiv!


%%
%% ----------------------------------------------------------
%%
\part{Form und Struktur}

\section{Bestandteile einer Projektarbeit}
Eine Projektarbeit enthält viele Bestandteile wie Kapitel, Abschnitte und Tabellen.
Zu jedem dieser Bestandteile zeigen die folgenden Unterabschnitte einige Tipps.

\subsection{Titelseite}
Die Titelseite enthält Titel, Autor, Jahr, Firmenlogo, Berufsbezeichnung 
bzw.\ Ausbildungsziel und Abgabetermin.

Obwohl es einfach aussieht: Eine gute Titelseite zu gestalten ist schwierig. 
Eine Empfehlung für eine Titelseite lautet:

\begin{itemize}
 \item Der Titel ist die wichtigste Information, deshalb sollte er möglichst auffällig sein.
  Entweder durch Schriftgröße oder Fettschrift.
 \item Vermeide Textauszeichnungen wie Sperren, zuviel Farbe oder Unterstreichungen.
 \item Halte dein Logo klein. Übergroße Grafiken lenken vom Titel ab. 
\end{itemize}

Ein möglicher Aufbau einer Titelseite ist in Abbildung~\ref{fig.titleseite}
dargestellt.

\begin{figure}[tb]
 \caption{Aufbau einer Titelseite\label{fig.titleseite}}
\framebox[.25\linewidth][c]{%
\begin{minipage}[t]{.5\textwidth}
%\begin{verbatim}
\begin{center}
\hspace{0pt}\\[1.125em]

          Autor\\[2ex]

        Der Titel\\
       (Untertitel)\\[2ex]

     Ausbildungsziel\\
      Abgabetermin\\[2ex]

          Logo\\[2ex]
\hspace{0pt}
\end{center}
%\end{verbatim}
\end{minipage}}
\end{figure}



\subsection{Inhaltsverzeichnis}
Das Inhaltsverzeichnis ist eine Übersicht deiner Projektarbeit, kein Index.
Aus diesem Grund reichen Einträge für Kapitel und Abschnitt völlig aus.
Unterabschnitte blähen das Inhaltsverzeichnis unnötig auf.
Verwende in \LaTeX\ den Befehl \verb!\tableofcontents!.


%\subsection{Erklärung}
% Jede Projektarbeit enthält eine Erklärung, dass du diese Arbeit selbstständig
% geschrieben hast.



\subsection{Vorwort}
In vielen Fällen wird das Vorwort weggelassen. Falls der Platz ausreicht
findest du folgendes:

\begin{itemize}
 \item Wenn es passt, schafft ein Zitat die nötige Atmosphäre.
 \item Typografische Konventionen. Dieser Abschnitt ermöglicht dem Leser
  einen schnellen Einstieg in deine \gquote{typografische Welt}. 
% \item Möglicherweise etwas zu dir als Autor. Falls du beispielsweise früher einen
% Webserver betreut hast und in deiner Projektarbeit dieses Thema bearbeitest.
 \item Danksagungen. Wer dir bei deiner Projektarbeit geholfen hat.
\end{itemize}



\subsection{Die Hauptstruktur}
Die oberste Struktur einer Projektarbeit ist meistens gleich aufgebaut. 
Lediglich bei Anhängen sind Variationen möglich. Häufig findet sich der 
komplette Quellcode dort. Aber auch umfangreichere Hintergründe, die 
am Anfang störend wären, lassen sich in den Anhang verschieben.

Die erste Hierarchie einer Projektarbeit enthält die folgenden Kapitel:

\begin{description}
 \item[1. Projektauftrag:] beschreibt das Umfeld, die aktuelle Situation und das Projektziel.
 \item[2. Projektplanung:] erklärt die Vorbedingungen, Zeiteinteilung, Kosten, Ressourcen u.~ä.
 \item[3. Projektdurchführung:] betrachtet Ablauf und Implementierung.
 \item[4. Projektabschluss:] enthält eine Zusammenfassung, Fazit oder Ausblick.
\end{description}



\subsection{Abschnitte}
Abschnitte ordnen den Text eines Kapitels oder Anhangs. Die folgende Liste fasst
die wichtigsten Punkte zusammen:

\begin{itemize}
 \item Strukturiere die Projektarbeit mit Abschnitten und Unterabschnitten. 
 \item Füge zwischen Abschnitt und folgendem Unterabschnitt einen einführenden
 Text oder eine kurze Zusammenfassung ein.
 \item Formuliere die Titel von Abschnitten so, dass sie \gquote{parallel} aufgebaut sind.
 Parallel sind Abschnitte dann, wenn alle gleichartig beginnen. Dies kann mit 
  einem Verb oder Substantiv sein.
 \item Vermeide Abschnitte die nur ein oder zwei Sätze enthalten. 
 \item Fasse mehrere Abschnitte mit wenigen Sätzen in einer Liste zusammen.
 \item Vermeide Abschnitte die nur einen einsamen Unterabschnitt enthalten.
\end{itemize}

\subsection{Glossar}
Ein Glossar, oder Fachwortverzeichnis, ist meist sinnvoll, wenn die Zielgruppe
unterschiedlich ist oder Begriffe in der Literatur uneinheitlich verwendet 
werden. Bei uneinheitlichen Begriffen klärt ein Glossar auf, wie der Begriff
in deiner Projektarbeit zu verstehen ist. Hierzu einige Tipps:

\begin{itemize}
 \item Verwende ein Glossar, wenn erhebliches Vorwissen vorausgesetzt wird
oder ein Begriff mit einem Wort schwer zu erklären ist. So kann der Leser im
 Glossar nachschlagen.
 \item Sortiere die Begriffe alphabetisch.
 \item Schreibe deine Erklärungen so, dass möglichst wenig Vorwissen benötigt wird.
 \item Achte darauf, dass alle Einträge entweder in der Einzahl oder Mehrzahl 
 geschrieben sind.
 \item Formatiere das Glossar als Liste, nie als Tabelle mit Linien.
\end{itemize}


\subsection{Literaturverzeichnis}
Jeder Text baut auf der Arbeit von Anderen auf. Falls du Literatur oder Seiten
im Web verwendest oder darauf verweist, nenne deine Quellen in einem 
\emph{Literaturverzeichnis}.
Beispiel~\ref{exa.biblio} zeigt, wie in \LaTeX{} ein Literaturverzeichnis von 
Hand eingegeben wird.

\begin{Example}[Manuell gesetztes Literaturverzeichnis in \LaTeX{}]\label{exa.biblio}
\vspace*{-.75em}
\begin{verbatim}
\begin{thebibliography}{baum08}
 \bibitem[Baum08]{bib.baumert} Andreas Baumert. 
  \emph{Professionell texten}. 
  Grundlagen, Tipps und Techniken. 2.~Auflage. dtv 2008. 
  ISBN 978-3-423-50868-1.
 % ... Weitere Einträge mit \bibitem[Label]{Citekey}
\end{thebibliography}
\end{verbatim}
\end{Example}

\begin{merksatz}
Überlege dir ein Schema, damit jeder Eintrag den gleichen Aufbau hat.
Beispielsweise Autor, Titel, Untertitel, Auflage, Verlag, Jahr, ISBN-Nummer.
Möglicherweise benötigst du weitere Einträge oder andere sind überflüssig
(Stichwort Webseite). 
\end{merksatz}


\subsection{Tabellen}
Tabellen sind ein großartiges Werkzeug, um Daten darzustellen. Achte auf die
folgenden Punkte:

\begin{itemize}
 \item Verwende Tabellen um Daten darzustellen, nicht um zweispaltigen Satz zu simulieren.
 \item Gib jeder Tabelle eine Tabellen\emph{überschrift}.
 \item Beschrifte jede Spalte.
 \item Verweise im Text auf die Tabelle.
 \item Setze Einheiten in einer Spalte als Tabellenfußnote, im Tabellenkopf oder in
 der Zelle und sei konsistent im gesamten Dokument.
 \item Achte auf Konsistenz hinsichtlich Linien und Schattierungen.
 \item Formatiere den Zelleninhalt je nach Anforderung link- oder rechtsbündig, jedoch 
   \emph{nie} im Blocksatz.
 \item Formatiere den Tabellenkopf anders als den Rest, beispielsweise fett oder kursiv.
\end{itemize}


\subsection{Abbildungen}
Eine Abbildung sagt mehr als tausend Worte, lautet ein Sprichwort. Zu Abbildungen
gibt es folgendes zu sagen:

\begin{itemize}
 \item Verwende Grafiken nur für Screenshots oder Zeichnungen, nicht für Quellcode.
  Anders als Grafiken lässt sich Text kopieren und durchsuchen.
 % Tipp für \LaTeX-Anwender: Empfohlen wird das Paket \latexenv{listing}.
 % Grund: Text kann später aus PDF nicht mehr kopiert werden, 
 % Suche funktioniert nicht mehr
 % Änderungen werden schwieriger
 % Umbruch
 \item Gib jeder Abbildung einen Titel \emph{unterhalb} des Bildes.
 \item Verwende das richtige Format: Bitmaps für Grafiken, Vektorformate für Zeichnungen.
 \item Wenn die Grafik ein Koordinatensystem hat, beschrifte \emph{immer} die Achsen, 
 \item Füge bei komplizierten Grafiken eine Legende ein und beschreibe den Zusammenhang 
      im Text.
 \item Beschneide die Grafik, wenn nur ein kleiner Teil interessiert. Alternative:
  Hervorheben durch Kreis oder Rechteck.
 \item Bevorzuge Grafiken mit größerer Breite als Höhe. Schlanke Grafiken können links
  oder rechts durch Text aufgefüllt werden, erschweren aber möglicherweise den Umbruch.
 \item Halte deine Grafiken konsistent hinsichtlich Schriftart, Strichstärke, 
  Farbe und Aufbau.
\end{itemize}



\subsection{Listen}
Listen sind sehr gut geeignet, um etwas aufzuzählen oder eine Reihenfolge anzugeben. 
Folgende Hinweise treffen auf Listen zu:

\begin{itemize}
 \item Beginne die Liste mit einen einleitenden Satz (wie im vorigen Satz gezeigt).
 \item Verwende eine nummerierte Liste (in \LaTeX\ \latexenv{enumerate}), wenn jeder
Schritt auf den anderen folgt und nicht ausgetauscht werden kann.
 \item Verwende eine Aufzählungsliste (in \LaTeX\ \latexenv{itemize}), wenn 
  es keine Reihenfolge zwischen den Listeneinträgen gibt. Jeder Eintrag kann
  mit einem anderen vertauscht werden, ohne dass es Probleme gibt.
 \item Stelle sicher, dass deine Liste minimal zwei Einträge hat.
 \item Begrenze deine Liste auf maximal 10 Einträge.
 \item Halte die Listeneinträge \gquote{parallel}, ähnlich wie bei den Abschnitten; 
  achte auf Einzahl und Mehrzahl.
 \item Beende die Listeneinträge nur bei ganzen Sätzen mit einem Punkt, nicht
  wenn sie aus einzelnen Wörtern bestehen.
\end{itemize}


\subsection{Fußnoten}
Manche Projektarbeiten verwenden Fußnoten. Dazu ist folgendes zu sagen:

\begin{itemize}
 \item Füge die Fußnote als \emph{Ergänzung}\footnote{Wie in diesem Beispiel gezeigt.}
  ein, nicht als zusätzliche Erklärung. Fußnoten sind \emph{optionale} Texte, denen der 
  Leser folgen \emph{kann} aber nicht muss. 
 \item Aus dem ersten folgt: Definiere Begriffe die der Leser verstehen muss im
  Text und \emph{nicht} in der Fußnote.
 \item Setze die Fußnote direkt hinter\footnote{Genau so und ohne Leerzeichen!}
  den Begriff den du erklären willst.
 \item Setze die Fußnote hinter das Satzende (Punkt), wenn du den ganzen Satz
  kommentieren möchtest.\footnote{Ein schöner Satz, nicht wahr? ;-)}
\end{itemize}



\section{Typografiesünden}
\begin{epigraph}{Hans-Peter Willberg}
 Typografie ist keine Wissenschaft. Typografie ist Handwerk.
\end{epigraph}
% \begin{epigraph}{Hans-Peter Willberg}
%  Man sollte nicht sein ganzes Leben nur mit Buchstaben verbringen,\\
%  es gibt ja noch die Typografie!
% \end{epigraph}


% \section{Typografie-Sünden vermeiden}

Die häufigsten Sünden sind ziemlich einfach zu vermeiden. Die folgenden
Abschnitte zeigen, wie du es richtig machst.

\subsection{Zu viele Schriftarten}
Für Projektarbeiten sind drei Schriftarten völlig ausreichend. Folgende 
Aufteilung hat sich bewährt:

\begin{itemize}
 \item Serifenschrift: für gewöhnlich als Textschrift.
 \item \textsf{Sanserifschrift}: meist für Überschriften
 \item \verb!Dicktengleiche Schrift! (engl.~\emph{monospace}): für Code, URLs, Dateinamen usw.
\end{itemize}


\subsection{Falsche Anführungszeichen}
Im Deutschen gibt es zwei Möglichkeiten Anführungszeichen zu verwenden:

\begin{itemize}
 \item \gquote{Deutsche Anführungszeichen}; Merksatz: 99 unten, 66 oben (Unicode U+201E und U+201C)
 \item \fquote{Französische Anführungszeichen}
\end{itemize}

Welche Form du verwendest ist eine Geschmacksfrage, beide sind üblich. Wichtig ist, 
dass du es im gesamten Dokument \emph{konsistent} machst.

Laut Duden werden Anführungszeichen (auch \gquote{Gänsefüßchen}) in
folgenden Situationen verwendet:

\begin{itemize}
 \item wörtliche Rede: Sie sagte: \gquote{Das Programm ist toll.}
 \item ironische Hervorhebung (so genanntes \gquote{typografisches Zwinkern}):\\
 Das Programm ist \gquote{ziemlich} stabil.
 \item für Fremdtexte, Zitate, Buch- und Filmtitel, Musikstücke und Ähnliches:\\
 Das Buch \gquote{Einführung in XML} stammt von Erik T.\ Ray.
\end{itemize}

\LaTeX{}-Anwender definieren folgende Befehle:

\begin{verbatim}
% Deutsche Anführungszeichen:
\newcommand{\gquote}[1]{\glqq #1\grqq}
% Französische Anführungszeichen:
\newcommand{\fquote}[1]{\frqq #1\flqq}
\end{verbatim}


\subsection{Auslassungspunkte}
Drei Punkte bedeuten, dass etwas ausgelassen worden ist (\LaTeX: \verb!\ldots!, Unicode U+2026):

\begin{Example}[Auslassungspunkte]
\hspace{0pt}\\[-1em]
\begin{tabular}{ll}
 \coltitle{Falsch}  & \coltitle{Richtig} \\
Verd...!            & Verd\ldots! \\
Viele... Entwickler & Viele\ldots\ Entwickler\\[1.5em]
\end{tabular} 
\end{Example}


\subsection{Verstümmelter Gedankenstrich}
Laut Duden wird der Gedankenstrich dort verwendet, \gquote{wo man in der
gesprochenen Sprache eine deutliche Pause macht}.
Meist können Gedankenstriche durch andere Satzzeichen wie Kommas oder 
Klammern ersetzt werden. Verwende in \LaTeX{} die Zeichen \verb!~--! (Unicode U+2013):


\begin{Example}[Gedankenstriche]
%\hspace{0pt}\\[-1em]
\begin{tabular}{ll}
 \coltitle{Falsch}  & \coltitle{Richtig} \\
XML - das ASCII des Webs. & XML~-- das ASCII des Webs. \\
\multicolumn{2}{l}{\LaTeX{}-Eingabe: \texttt{XML\textasciitilde-{}- das ASCII des Webs.}}
\end{tabular}
\end{Example}


\subsection{Verschwenderische Auszeichnungen}
Text lässt sich auf verschiedene Art auszeichnen:

\begin{itemize}
 \item als \textbf{Fettschrift}
 \item \textit{kursiv} (vgl.\ \textit{kursiv} und \textsl{kursiv}; 
   das Zweite ist nur schräggestellt)
 \item \textsf{durch Wechsel der Schriftart}
 \item als \textsc{Kapitälchen}
 \item \underline{unterstrichen}
 \item als in GROSSBUCHSTABEN (sog. Versalien)
 \item g\,e\,s\,p\,e\,r\,r\,t
 \item in \textcolor{red}{roter Farbe}, als \colorbox{light}{unterlegter Text}
 \item durch Einrückungen (eher für ganze Sätze geeignet)
\end{itemize}

In einer Projektarbeit sind gängige Auszeichnungen: \textbf{fett}, \textit{kursiv},
\textsc{Kapitälchen} oder eine andere \textsf{Schriftart}. Andere Auszeichnungen
sollten vermieden werden, da sie entweder veraltet sind (unterstreichen), zuviel 
Aufmerksamkeit erregen (GROSSBUCHSTABEN), stümperhaft aussehen 
(s\,p\,e\,r\,r\,e\,n) oder Probleme beim Ausdruck ergeben (Farbe).


\appendix
\section{Bewertungsschema}
Laut Sandra Farrell schaut die IHK auf folgende Punkte:

\begin{itemize}
 \item Gestaltung des prozessorientierten Projektberichts
 \item Gliederung angemessen (ersichtliche Gedankenführung, passende Detaillierung)
 \item Überschriften (aussagekräftig, knapp und dennoch klar)
 \item Quellennachweis, Anlagenverzeichnis, Abkürzungsverzeichnis, Glossar,
       Literaturhinweis (Notwendigkeit, qualitativ sinnvolle Angaben)
 \item Hinweise und Erläuterungen zu den beigefügten praxisbezogenen Unterlagen
 \item Formale Gestaltung (Lesbarkeit, Randgestaltung, Zeilenabstände, Seitenangaben,
       Schriftart, durchgehende übersichtliche Nummerierung, Visualisierungen\ldots)
 \item Sprachliche Gestaltung (sachliche und flüssige Sprache, Rechtschreibung,
       Verständlichkeit)
\end{itemize}



\section{Abschließende Tipps/Zusammenfassung}
Folgender Ablauf hat sich bei mir bewährt:

\begin{enumerate}
 \item Lies deinen Text laut. Dadurch erkennst du Probleme in deinem Text.
 \item Streiche Füllwörter und überflüssige Adjektive. 
 \item Streiche alle Stellen an, die seltsam klingen oder dir nicht gefallen.
 \item Prüfe den logischen und \gquote{dramaturgischen}  Ablauf.
% \item 
 \item Überarbeite alle angestrichenen Stellen.
 \item Lass Andere deinen Text lesen.
 \item Überarbeite die Passagen, die anderen Lesern missfallen haben.
 \item Lies nochmals laut vor.
 \item Lass den Text einige Tage liegen und wiederhole die Schritte bei Bedarf.
\end{enumerate}




%%
%% Literaturverzeichnis
%%
\begin{thebibliography}{baum08}
 \bibitem[Baum08]{bib.baumert} Andreas Baumert. \emph{Professionell texten}. Grundlagen, Tipps und
Techniken. 2.~Auflage. dtv 2008. ISBN 978-3-423-50868-1.
 \bibitem[Ros05]{bib.rosenberg} Barry J.\ Rosenberg. 
\emph{Spring into Technical Writing for Engineers and Scientists}.
Addison-Wesley 2005. ISBN 0-13-149863-0.
 \bibitem[Schul05]{bib.schulze} Hans Herbert Schulze. \emph{Computer-Englisch}.
Ein englisch-deutsches und deutsch-englisches 
Fachwörterbuch. rororo 2005. ISBN 3-499-61260-7.
 \bibitem[Schn01]{bib.schneider01} Wolf Schneider. \emph{Deutsch für Profis}.
Wege zum guten Stil. Mosaik bei Goldmann 2001. ISBN 3-442-16175-4.
 \bibitem[Schn05]{bib.schneider05} Wolf Schneider. \emph{Deutsch für Kenner}.
Die neue Stilkunde. Piper Verlag 2005. ISBN 3-492-24461-0.
 \bibitem[Schn06]{bib.schneider06} Wolf Schneider. \emph{Wörter machen Leute}.
Magie und Macht der Sprache. Piper Verlag 2006. ISBN 3-492-20479-1.
\end{thebibliography}

%\appendix

\end{document}
\endinput

Dank der redaktionseigenen Glaskugel präsentiert heise Security schon jetzt und 
wie immer exklusiv den ultimativen Jahresrückblick auf das vor uns liegende Jahr 2010.

Unwort des Jahres 2009
http://www.welt.de/kultur/article5893478/Das-Unwort-des-Jahres-2009-steht-fest.html
