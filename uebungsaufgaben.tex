%
%
%
%\documentclass[a4paper,titlepage=true,twoside]{scrartcl}
\documentclass[a4paper,twoside]{article}
\usepackage[utf8x]{inputenc}
\usepackage[ngerman]{babel}
\usepackage{setspace}

\usepackage[%
forcolorpaper,%
forpaper,%
%answerkey,%
%nopoints,%
%solutionsafter,%
]{eqexam}

%opening
\title{Schreiben einer Projektarbeit}
%\subtitle{Struktur, Sprache, Typografie}
\author{Thomas Schraitle\\\email{toms@suse.de}}
\subject{Übungsaufgaben}
\keywords{Sprache}
\university{SUSE Linux Products GmbH}
\date{\today\\Version 1.0}

%\title{}
%\author{}

\defaultInstructions{Einleitung.}
%\examNum{2}
\forVersion{B}
\eqCommentsColor{blue}
\instructionsColor{blue}

\def\exsectitle{\normalsize\hspace*
    {-\oddsidemargin}L\"osungen zum \webtitle}
\def\exsecrunhead{L\"osungen zum \websubject}%

\newcommand{\gquote}[1]{\glqq #1\grqq}


%%
%% ---------------------------------------------
\begin{document}

\maketitle

\begin{exam}[Teil I.]{Part1}
\begin{instructions}[Wörter]
Einige Übungen zu Wörtern.
\end{instructions}

\begin{problem}[3]
Übersetze folgende Hauptwörter in Verben:\\
eine Entscheidung treffen, % entscheiden;
eine Programmierung vornehmen, % programmierung
eine Untersuchung durchführen, % untersuchen;
Schaden zufügen, % schaden;
in Augenschein nehmen, % anschauen;
Abhilfe schaffen, % abhelfen, lösen;
einer Prüfung unterziehen, % prüfen;
zur Anwendung kommen, % anwenden;
in Betrieb nehmen, % einschalten, betreiben;
Anwendung finden, % anwenden;
die Steuerung übernehmen, % steuern;
einer Kontrolle unterwerfen, % kontrollieren;
zur Verfügung stellen, % geben, anbieten;
\begin{solution}
eine Entscheidung treffen = entscheiden;
eine Programmierung vornehmen = programmieren;
eine Untersuchung durchführen = untersuchen;
Schaden zufügen = schaden;
in Augenschein nehmen = anschauen;
Abhilfe schaffen = abhelfen, lösen;
einer Prüfung unterziehen = prüfen;
zur Anwendung kommen = anwenden;
in Betrieb nehmen = einschalten, betreiben;
Anwendung finden = anwenden;
die Steuerung übernehmen = steuern;
einer Kontrolle unterwerfen = kontrollieren;
zur Verfügung stellen = geben, anbieten, präsentieren;
\end{solution}
\end{problem}


\begin{problem}[3]
Übersetze folgende englische Wörter ins Deutsche:\\
%\begin{doublespacing}
approval, % Genehmigung
application, % Anwendung
burner, % Brenner
button, % Knopf, Schaltfläche, Taste
exception handling, % Ausnahmebehandlung
icon, % Bildsymbol, Piktogramm
check list, % Zusammenstellung, Kontroll-Liste
code of conduct, % ethischer Verhaltenscode
conformance test, % Übereinstimmungstest
download, % herunterladen
font, % Schriftart
hyphenation, % Silbentrennung
level, % Ebene, Niveau, Pegel, Stufe
memory card, % Speicherkarte
memory dump, % Speicherauszug
navigation bar, % Navigationsleiste
operating system, % Betriebssystem
preselection, % Vorauswahl
right-justified, % rechtsbündig
refactor/refactoring, % Überarbeiten
shareware, % Software, die erst nach kostenloser Prüfung bezahlt wird
software, % Programm
software auditing, % Software-Qualitätsprüfung
soft link, % symbolische Verknüpfung
update, % aktualisieren, ändern, auf den neuesten Stand bringen
upgrade, % aufrüsten, ausbauen, erweitern, umsteigen
wizard, % Assistent, Hilfefunktion, Experte
word spacing, % Wortzwischenraum
%\end{doublespacing}
\begin{solution}
approval = Genehmigung, Bewilligung, Befürwortung, Billigung, Abnahme;
application =  Anwendung;
burner =  Brenner;
button =  Knopf, Schaltfläche, Taste;
exception handling =  Ausnahmebehandlung, Ausnahme, Fehler;
icon =  Bildsymbol, Bildzeichen, Piktogramm;
check list =  Zusammenstellung, Kontroll-Liste;
code of conduct =  (ethischer) Verhaltenskodex, Verhaltensregeln, Ehrenkodex, Leitfaden;
conformance test =  Übereinstimmungstest, Test auf Übereinstimmung;
download =  herunterladen;
font =  Schriftart;
hyphenation =  Silbentrennung;
level =  Ebene, Niveau, Pegel, Stufe;
memory card =  Speicherkarte;
memory dump =  Speicherauszug;
navigation bar =  Navigationsleiste;
operating system =  Betriebssystem;
preselection =  Vorauswahl;
right-justified =  rechtsbündig;
refactor/refactoring =  Überarbeiten/Überarbeitung;
shareware =  Software, die erst nach kostenloser Prüfung bezahlt wird;
software =  Programm;
software auditing =  Software-Qualitätsprüfung;
soft link =  symbolische Verknüpfung;
update =  aktualisieren, ändern, auf den neuesten Stand bringen;
upgrade =  aufrüsten, ausbauen, erweitern, umsteigen;
wizard =  Assistent, Hilfefunktion, Experte;
word spacing = Wortzwischenraum
\end{solution}
\end{problem}


\begin{problem}[3]
Finde Erklärungen zu folgenden englischen Begriffen:\\
applet, % kleine Anwendung; spez. eingebettete Funktion in komplexen Anwendungen
API, % API, Anwendungsprogramm-Schnittstelle
%byte code, %
clip art, % Sammlung von Grafikbildchen
eye tracking, % 
test case, % Testumgebung

\begin{solution}
Applet = kleine Anwendung oder spez.\ eingebettete Funktion in komplexen Anwendungen;
API = API, Programmschnittstelle;
%byte code = vorübersetzte ;
clip art = Sammlung von Grafikbildchen;
eye tracking = Augenverfolgung, Bewegungserfassung der Augen;
test case =  Testumgebung;
widget = (window + gadget)  Vorrichtung im Fenster, Dingsbums, Komponente;
\end{solution}
\end{problem}

% \begin{instructions}[Zielgruppe]
% 
% \end{instructions}
% 
% \begin{problem}[3]
% Beantworte die Frage: Was ist ein Paketmanager? Wie sieht deine Antwort aus,
% wenn du sie folgenden Personen mitteilst?
% \begin{itemize}
%  \item Einem neuen Kollegen, der frisch in deiner Abteilung angefangen hat
%  \item Einem Freund, der Windows verwendet
%  \item Deiner Oma
% \end{itemize}
% 
% \begin{solution}%[1.5em]
% Kollege:
% Windows-Nutzer:
% Oma:
% \end{solution}
% \end{problem}



\begin{instructions}[Sätze]
Folgende Sätze sind aus verschiedenen Gründen problematisch. Warum? Wie lassen
sich die Probleme vermeiden?
\end{instructions}

\begin{problem}[3]
Das Geben der Milch passiert durch Kühe, und das Legen der Eier
geschieht durch Hühner.
\begin{solution}
Problem: Substantivstil\\
Kühe geben Milch und Hühner legen Eier.
\end{solution}
\end{problem}


\begin{problem}[3]
Zwei Tote in Schöpperstedt\\
71-jähriger soll sich und später seine 65-jährige Ehefrau erschossen haben.
\begin{solution}%[1.5em]
Problem: Reihenfolge\\
71-jähriger soll seine 65-jährige Ehefrau und [dann] sich selbst erschossen haben.
\end{solution}
\end{problem}

\begin{problem}[3]
Die Qt-Programmiersprache stellt verschieden Module zur Verfügung, um
die Plattformunabhängigkeit von Programmen zu bewerkstelligen.
\begin{solution}%[1.5em]
Problem: Qt ist keine Programmiersprache, sondern eine Bibliothek/Toolkit; 
zur Verfügung stellen; Plattformunabhängigkeit\\
Die Qt-Bibliothek enthält verschiedene Module, um plattformunabhängig zu programmieren.
\end{solution}
\end{problem}

\begin{problem}[3]
Die Programmiersprache Perl besticht durch Plattformunabhängigkeit, Freiheit und ist eine 
Skriptsprache, die durch einen Interpreter interpretiert wird.
\begin{solution}%[1.5em]
Problem: Substantivstil, Verdopplungen (Interpreter interpretiert)\\
Perl ist eine freie, plattformunabhängige und interpretierte 
Programmiersprache (Skriptsprache).
\end{solution}
\end{problem}

\begin{problem}[3]
Das Dokumentfenster, das unter dem Fenster, das zu dem Diskettensymbol gehört, 
liegt, müssen Sie, bevor Sie das Programm beenden, schließen.
\begin{solution}
Problem: Schachtelsatz\\
Bevor Sie das Programm beenden, müssen Sie das Dokumentfenster schließen.
Das Dokumentfernster liegt unter dem Fenster, das zum Diskettensymbol gehört.
\end{solution}
\end{problem}



\begin{problem}[3]
Das Handbuch ist zur Vermeidung von Störungen oder Schäden beim Betrieb
zu beachten und daher vom Betreiber dem jeweiligen Wartungs- und Bedienpersonal
zur Verfügung zu stellen.
\begin{solution}
Problem: Substantivstil\\
Um Störungen oder Schäden beim Betrieb zu vermeiden beachten Sie das Handbuch
und geben Sie dieses dem jeweiligen Wartungs- und Bedienpersonal.
\end{solution}
\end{problem}

\begin{problem}[3]
Beim Schließen des letzten Fensters wird automatisch das Programm beendet.
\begin{solution}
Problem: Passiv, unklar \emph{wer} etwas macht.\\
Wenn Sie das letzte Fenster schließen, dann beendet der Computer das
Programm automatisch.
\end{solution}
\end{problem}

\begin{problem}[3]
Darüberhinaus müssen Perl-Skripte nicht compiliert werden; dadurch wird die 
benötigte Zeit für die Entwicklung der Anwendung verringert.
\begin{solution}
Problem: \gquote{dadurch} passt nicht\\
Perl-Skripte werden von einem Interpreter [sofort] ausgeführt, nicht übersetzt. 
Die eingesparte Zeit beim Übersetzen verringert die Entwicklungszeit.
\end{solution}
\end{problem}

\begin{problem}[3]
Bitte legen Sie die Handtücher, die Sie nicht mehr benötigen, in Ihr Waschbecken.
Wir werden Sie dann austauschen.
\begin{solution}
Problem: Was ist mit \gquote{Sie} gemeint: Tippfehler, Handtücher oder Kunde?\\
Lösung 1: Bitte legen Sie die Handtücher, die Sie nicht mehr benötigen, in Ihr Waschbecken.
Wir werden die Handtücher dann austauschen.\\
Lösung 2: Nicht mehr benötigte Handtücher werden ausgetauscht, wenn Sie diese
in das Waschbecken legen.
\end{solution}
\end{problem}

\begin{problem}[3]
Wenn man sich also entschieden haben sollte, dass man auf ein Wacom Grafik-Tablett
setzen möchte, dann kann durch die Installation des Treibers xyz diesen
zur Funktion gebracht werden.
\begin{solution}
Problem: Ausdruck, Füllwörter, Nominalstil, überflüssige Redewendung\\
Hat man sich/Haben Sie sich für ein Wacom-Grafik-Tablett entschieden, installieren
Sie den Treiber xyz um ihn zu aktivieren.
\end{solution}
\end{problem}

\begin{problem}[3]
Die Erfassung der Wartezeit, die ein Kunde in der Queue¹ auf die Annahme seines
Kundengesprächs gewartet hat, soll nach Möglichkeit auch implementiert werden.
\begin{solution}
Problem: \gquote{Queue}, Substantivstil, Reihenfolge und Einschub\\
(Als) ein weiteres Ziel soll die Zeit erfasst werden, die ein Kunde 
[in der Warteschlange] wartet bis sein Gespräch durch einen Kundenberater
angenommen wird.
\end{solution}
\end{problem}

\begin{problem}[3]
Da Foo 2.0 plant eine XML Konfiguration zu verwenden, um das Erstellen und 
Auswerten für weitere Applikationen zu erleichtern, sollte Food mit einem 
XML Schema (»XSD«) das Konfiguration Layout bestimmen. Somit kann die zu
erstellende Codebasis in Foo 2.0 wiederverwendet werden.
\begin{solution}
Problem: Zu lange Sätze, Reihenfolge\\
Um das Erstellen und Auswerten für weitere Anwendungen zu erleichtern, plant
das Foo-Projekt in Version 2.0 eine XML-Konfiguration zu verwenden. Das
Programm Food kann durch diese XML-Konfiguration auf Basis eines XML Schema 
(XSD) konfiguriert werden und somit die vorhandene Codebasis wiederverwenden.
\end{solution}
\end{problem}

\begin{problem}[3]
Um die Code Qualität auch in Zukunft von Food möglichst hoch halten zu können,
ist es dringend erforderlich neben guter Programmcode Dokumentation eine Vielzahl
an Testcases anzubieten.
\begin{solution}
Problem: Ausdruck von \gquote{hoch halten}, \gquote{dringend erforderlich}
Um die Qualität des Quellcodes auch in Zukunft sicherzustellen, werden zwei
Methoden verwendet: Dokumentieren des Quellcodes und eine Vielzahl von Tests.
\end{solution}
\end{problem}

\begin{problem}[3]
Da Food als Session Daemon ständig in Betrieb ist, ist es unabdingbar,
dass der Programmcode auf Speicherlecks hin überprüft wird. 
\begin{solution}
Problem: Ausdruck \gquote{unabdingbar}
Das Programm Food läuft permanent im Hintergrund. Dadurch ist es erforderlich,
den Quellcode auf Speicherlecks zu überprüfen.
\end{solution}
\end{problem}


\begin{problem}[3]
Neben nicht trivial (re?)produzierbaren Fehlern durch die Benutzung von
XML Bibliothek »libxml2«, sind Probleme die beim allokieren von Speicher 
auftreten können nicht durch die Unittests abgedeckt. 
\begin{solution}
Problem: Zuviel Information auf einmal\\
Wird die XML-Bibliothek libxml2 verwendet, können nicht-reproduzierbare
Fehler auftreten. 
Die Überprüfung des Quellcodes enthält keine Testfälle um Probleme
beim Anfordern von Speicherplatz zu entdecken.
\end{solution}
\end{problem}


\begin{problem}[3]
FooC sowie FooS sind gegen statische Bibliotheken aus dem
Synergy-Projekt gelinkt. 
\begin{solution}
Problem: Ausdruck \gquote{gelinkt}
Die Programme FooC und FooS verwenden Bibliotheken aus dem Synergie-Projekt.
\end{solution}
\end{problem}


\begin{problem}[3]
Ein in der Bundespolizeiakademie in Lübeck modifiziertes Gerät, das 
Intimbereiche von Flugpassagieren pixelt, wird laut Focus im Januar dem 
Staatssekretär im Bundesinnenministerium, Klaus Dieter Fritsche, vorgeführt.
\begin{solution}
Problem: Ausdruck \gquote{pixelt}
In der Bundespolizeiakademie [in Lübeck] wurde ein Gerät dermaßen modifiziert, 
dass Intimbereiche von Flugpassagieren unkenntlich gemacht werden.
Laut Focus wird das Gerät im Januar von Klaus Dieter Fritsche, Staatssekretär
im Bundesinnenministerium, vorgeführt.
\end{solution}
\end{problem}


% \begin{problem}[3]
% 
% \begin{solution}
% \end{solution}
% \end{problem}

%\end{exam}

% \newpage
%\begin{exam}[Teil II.]{Part2}
\begin{instructions}[Korrektur]
Der folgende Text hat einige Schwächen. Identifiziere und verbessere den Artikel.
\end{instructions}

\begin{problem}[10]
Für die Benutzung von Linux wird ein Paketmanager gebraucht, mit der die
Installation von Paketen ermöglicht wird. Man muss nur aufpassen, dass sie
kompatibel zur Linux-Distribution wird. Übrigens wurde Linux 1992 erfunden,
von einem gewissen Linux Torvalds. Muss wohl Finne sein. So ein Paketmanager
ist dazu gedacht, einfach alle Pakete, für die du dich interessiert, auf
einen Schlag auf deine Platte zu hauen. Der überprüft dann auch, ob
alles kompatibel ist.
\begin{solution}
Erfunden wurde Linux 1992 von dem Finne Linus Torvalds. Um in Linux Pakete
zu installieren, benötigen Sie einen Paketmanager. Ein Paketmanager ist ein 
Programm, das überprüft, ob das Paket kompatibel zu Ihrer Version ist.
\end{solution}
\end{problem}


\begin{problem}[10]
** Die vollständig interaktive und leistungsfähige Benutzeroberfläche ist auch
für die Eingabe oder Modifzierung größerer Datenmengen gut geeignet. Diese
Schnittstelle ist optimal auf menschliche Faktoren und Leistungsparameter des
jeweiligen Anwendungsgebietes abgestimmt und erfordert keine systemspezifischen
Kenntnisse, sondern allein die fachliche Kenntnis der Funktionen, die im
Zusammenhang stehen mit dem Vermittlungsvorgang.
\begin{solution}
Ich habe für CARLA eine Benutzerobefläche geschaffen, die einem Formularblatt
auf Papier ähnelt. In den Eingabefeldern können Sie auch größere Datenmengen
problemlos eingeben oder ändern. Sie müssen keine komplizierten Befehle mehr
auswendig wissen. Sie klicken lediglich auf eines der vorgesehenen Sinnbilder
(Icons) oder wählen den Befehl aus dem Menü. Kurzum: Sie können sich ganz auf
die fachliche Seite der Aufgabe konzentrieren.
\end{solution}
\end{problem}


\end{exam}


% \begin{problem}[10]
%  Im Kinderanfall unserer Stadtgemeinde ist eine hierorts wohnhafte, noch
% unbeschulte Minderjährige aktenkundig, welche durch ihre unübliche
% gewohnheitsrechtlich R.~genannt zu werden pflegt. Der Mutter wurde seitens
% deren Mutter ein Schreiben zugestellt, in welchem diesselbe Mitteilung
% ihrer Krankheit und Pflegebedürftigkeit machte, worauf die Mutter der 
% R.~dieser die Auflage machte, der Großmutter eine Sendung von Nahrungs-
% und Genussmitteln zu Genesungszwecken zuzustellen.
% 
% Vor ihrer Inmarschsetzung wurde die R.~seitens ihrer Mutter schulisch
% über das Verbot betreffs Verlassens der Waldwege auf Kreisebene belehrt.
% Diesselbe machte sich infolge Nichtbeachtung dieser Vorschrift straffällig
% und begegnete beim Übertreten des diesbezüglichen Blumenpflückverbots einem
% polizeilich nicht gemeldenten Wolf ohne festen Wohnsitz. Dieser verlangte
% in unberechtiger Amtsanmaßung Einsichtnahme in
% das zu Transportzwecken von Konsumgütern dienende Korbbehältnis und traf
% in Tötungsabsicht die Feststellung, dass die R.~zu ihrer verschwägerten und
% verwandten, im Baumbestand angemieteten Großmutter eilends war.
% 
% Da wolfseits Verknappungen auf dem Ernährungssektor vorherrschend waren,
% fasste er den Beschluss, bei der Großmutter von R.~unter falscher Papiere
% vorsprachig zu werden. Weil diesselbe wegen Augenleidesn krank geschrieben
% war, gelang dem in Fressvorbereitung befindlichen Untier die diesfallsige
% Täuschungsabsicht, worauf es unter Verchlingung der Bettlägerigen einen
% strafbaren Mundraub zur Durchführung brachte.
% 
% Ferner täuschte das Tier bei später eintreffender R.~seine Identität mit 
% der Großmutter vor, stellte derselben nach und durch Zweitverschlingung 
% der R.~seinen Tötungsvorsatz erneut unter Beweis. Der sich auf dem Dienstweg
% befindliche und im Zuge der Beförsterung zuständige Waldbeamte B.~vernahm
% Schnarchgeräusche und stellte deren Urheberschaft seitens des Tiermaules fest.
% Er reichte bei seinen vorgesetzten Diensstelle ein Tötungsgesuch ein, das 
% dortsseits zuschlägig beschieden und pro Schuss bezuschusst wurde. Nach
% Beschaffung einer Pulverschießvorrichtung zu Jagdzwecken gab er in
% wahrgenommener Einflußnahme auf das Raubwesen einen Schuss ab.
% \end{problem}

\begin{exam}[Teil 2.]{Part2}
\begin{instructions}[Hausaufgaben]
Wähle aus den folgenden Themen \emph{eines} aus und schreibe einen kleinen
Artikel. Umfang ein bis zwei Seiten.
\end{instructions}

\begin{enumerate}
 \item Wie installiere ich openSUSE auf meinem Rechner?
 \item Beschreibe die Vorzüge des KDE/GNOME/\ldots-Desktops
 \item Was ist deine Lieblingsprogrammiersprache und warum?
 \item Wie verwalte ich meine Musiksammlung/Bilder unter openSUSE?
 \item Erkläre die Grundlagen von OpenOffice/\LaTeX/\ldots
 \item Wie konfiguriere ich einen Webserver?
\end{enumerate}

\end{exam}

\end{document}
